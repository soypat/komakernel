\usepackage[spanish, mexico]{babel}
\usepackage[utf8]{inputenc}
\usepackage[T1]{fontenc}
\usepackage{typearea}

 \usepackage[svgnames,x11names]{xcolor}


\usepackage{xspace}

%\usepackage{pifont}

%\usepackage{framed}
%\usepackage{wrapfig}
%\usepackage{enumerate}
%\usepackage{caption}
%\usepackage[spanish]{minitoc}  %Mini table of contents at the beginning of section/chapter
%\usepackage{mwe}  %Minimal working example

\usepackage[markcase=ignoreuppercase,headsepline,footsepline,plainfootsepline]{scrlayer-scrpage}
%\usepackage{hyperref}
%\hypersetup{pdfborder={0 0 0}}
\usepackage{bookman}

%\usepackage{capt-of}
%\usepackage{multicol}
%\usepackage[stable]{footmisc} %para footnote dentro de pma
%\usepackage{graphicx}
%\usepackage{setspace}
%\singlespacing
%\onehalfspacing

 %\usepackage{pgfplots}
 %\pgfplotsset{compat=1.15}
 %\addto\shorthandsspanish{\spanishdeactivate{>}}

\usepackage{tikz}
%\usetikzlibrary{calc,intersections,shapes.arrows,patterns}

%\usepackage{filecontents}
% %\usetikzlibrary{decorations.pathmorphing}
% \usetikzlibrary{shapes.geometric}
% \usetikzlibrary{calc}
% %\usetikzlibrary{arrows,snakes,shapes.geometric}
% 
% \usetikzlibrary{calc}
% \addto\shorthandsspanish{\spanishdeactivate{>}}
%\deactivatetilden

%\usepackage{mathtools}
%\usepackage{csquotes}
%\usepackage{hyperref}

%\usepackage{chngcntr}
%\usepackage{etoolbox}
%\usepackage{subfigure}

%
	\usepackage[backend=biber,
	    bibencoding=utf8,
	%    refsegment=chapter,
	    style=authoryear-ibid,
	    isbn=true,
	    doi=true,
	    url=false,
	    language=autobib,
	    clearlang=true,
	    firstinits=true,
	    defernumbers=true,
	    maxbibnames=5,
	    maxcitenames=2,minnames=1%,maxalphanames=4,minalphanames=3
	]{biblatex}
%	\DefineBibliographyStrings{spanish}{andothers={\textit{et~al\adddot}}}
%	\defbibheading{subbibliography}{%
%	                                 \section*{%
%	                                           Referencias y bibliografía
%	                                          }%                    
%	\addcontentsline{toc}{section}{Referencias y bibliografía}%
%	                               }
%	\newcounter{nAppendix}
%	\newcommand{\hAppendixSection}{\section*{Apéndices}{\addcontentsline{toc}{section}{\rmfamily Apéndices}}\setcounter{nAppendix}{0}}
%	\newcommand{\hBibliography}{\section*{Bibliografía y referencias}{\addcontentsline{toc}{section}{Bibliografía y referencias}}}
%	\newcommand{\hAppendix}[3]{\addtocounter{nAppendix}{1}%
%	                           \subsection*{{\thechapter}A.\Roman{nAppendix}. {#1}}%
%				    \label{a:#3}
%				    \addcontentsline{toc}{subsection}{{\thechapter}A.\Roman{nAppendix}. {#2}}}
%	\usepackage[mathlines,displaymath]{lineno}
%	%%%%%%%%%%%%%%%%%%%%%%%%%%%%%%%%%%%%%%%%%%%%%%%%%%%%%%%%
%	\newcommand*\patchAmsMathEnvironmentForLineno[1]{%
%	  \expandafter\let\csname old#1\expandafter\endcsname\csname #1\endcsname
%	  \expandafter\let\csname oldend#1\expandafter\endcsname\csname end#1\endcsname
%	  \renewenvironment{#1}%
%	     {\linenomath\csname old#1\endcsname}%
%	     {\csname oldend#1\endcsname\endlinenomath}}% 
%	\newcommand*\patchBothAmsMathEnvironmentsForLineno[1]{%
%	  \patchAmsMathEnvironmentForLineno{#1}%
%	  \patchAmsMathEnvironmentForLineno{#1*}}%
%	\AtBeginDocument{%
%	\patchBothAmsMathEnvironmentsForLineno{equation}%
%	\patchBothAmsMathEnvironmentsForLineno{align}%
%	\patchBothAmsMathEnvironmentsForLineno{flalign}%
%	\patchBothAmsMathEnvironmentsForLineno{alignat}%
%	\patchBothAmsMathEnvironmentsForLineno{gather}%
%	\patchBothAmsMathEnvironmentsForLineno{multline}%
%	}
