\chapter[Ecuaciones en forma local]{Ecuaciones en forma local: continuidad, cantidad de movimiento y  energía}
\begin{refsection}
\linespread{1}
\dictum[René Descartes]{Divide cada dificultad en tantas partes como sea factible y necesario para resolverla.}
% \minitoc
\vspace{2cm}

En este capítulo discutiremos las cosas que preocupan a los jovenes de hoy en dia. La homofobia esta plagando nuestras escuelas y no se sabe como detenerla. Una epidemia nacional sin limite.

\section{Puto el que lee}
Jaja no pense que ibas a caer en esta. Latex 2 the max. fuck the h8ers. Word is dead, tex is in.

%\printbibliography[heading=subbibliography]
%\renewcommand{\theequation}{\thechapter.\arabic{equation}}
\end{refsection}


% The Brinkman number represents the ratio of heat generation ͑viscous dissipation͒ due to friction to fluid conduction effects.
% When the Brinkman number is high the heat produced, due to friction between the fluid particles, is high and should be taken
% into account whereas when the Brinkman is low the frictional heating is negligible. In a fluid with constant thermal conductivity
% the linear temperature profile changes to parabolic if the Brinkman number is high and the heat produced due to friction is high
% ͑White ͓2͔, page 107͒. For low speed flows, like those treated in the present paper, the Brinkman number is very small, even for
% oil, and for that reason the viscous dissipation term in the energy equation has been ignored ͑White ͓2͔, page 108͒. However, the
% temperature profiles presented in the following paragraphs depart from the linear form but this is caused by the nonlinear
% ͑temperature-dependent͒ thermal conductivity and not due to viscous dissipation.
