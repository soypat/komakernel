\chapter[Indroducciónl]{Introducción a la termodinámica}
\begin{refsection}
\linespread{1}
\dictum[William Thomson, 1st Baron Kelvin acerca la segunda ley.
]{It is impossible, by means of inanimate material agency, to derive mechanical effect from any portion of matter by cooling it below the temperature of the coldest of the surrounding objects.}
% \minitoc
\vspace{2cm}

\section{Objetivo de la Termodinámica}
La Termodinámica es un capítulo de la Física y por ende su objetivo es el estudio de la naturaleza. Si bien sus resultados son completamente generales, la Termodinámica es particularmente útil cuando se la aplica al análisis de los sistemas compuestos por muchas partículas. Este curso trata sobre Termodinámica Clásica. En ella no se toma en cuenta la estructura de las partículas que forman la materia.

\section{Sistemas termodinámicos}
Sistema es toda porción de la naturaleza que se desea estudiar. Un sistema queda definido por una superficie cerrada, que se denomina frontera.

{\bf Sistema simple:} No contiene fronteras interiores

{\bf Sistema Compuesto:}



%\printbibliography[heading=subbibliography]
%\renewcommand{\theequation}{\thechapter.\arabic{equation}}
\end{refsection}
